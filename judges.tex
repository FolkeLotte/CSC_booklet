\chapter{Rencontrez les Juges}

\section*{John MacLeod}
\nopagebreak

\begin{wrapfigure}{r}{0.35\textwidth}
  \centering
  \includegraphics[width=0.33\textwidth]{Placeholders/ab87043f15a845cd1608e3085d710ce4.jpg}
\end{wrapfigure}
\color{red}
John MacLeod participe à des concours de chiens de berger depuis plus de quarante ans, travaillant dans des fermes de montagne à travers les Highlands écossais. Sa compréhension approfondie de l'élevage, du dressage et de la gestion du bétail a fait de lui l'une des voix les plus respectées du travail de chiens de berger en Europe. John a jugé des concours nationaux et internationaux dans sept pays et est connu pour ses évaluations justes et cohérentes qui récompensent les véritables capacités de travail plutôt que les tendances stylistiques. Il apprécie les chiens qui pensent de manière indépendante, s'adaptent rapidement aux situations difficiles et maintiennent une autorité calme avec le bétail. Ses propres chiens ont remporté de nombreux championnats, mais John reste surtout fier de ceux qui sont devenus des partenaires de travail fiables dans des fermes de montagne isolées. Pour lui, un grand concours de chiens de berger reflète les réalités de la vie quotidienne du berger—où la précision, la patience et la confiance comptent avant tout. John apporte des décennies d'expérience pratique à ce championnat et se réjouit de reconnaître l'excellence sur le terrain.
\color{black}

\newpage
\section*{Marie Dubois}
\nopagebreak

\begin{wrapfigure}{l}{0.35\textwidth}
  \centering
  \includegraphics[width=0.33\textwidth]{Placeholders/ab87043f15a845cd1608e3085d710ce4.jpg}
\end{wrapfigure}
\color{red}
Marie Dubois a grandi dans une ferme familiale des Pyrénées françaises, où travailler avec des chiens de berger n'était pas un sport mais une nécessité. Elle a commencé à participer à des concours à l'âge de seize ans et s'est rapidement imposée comme une conductrice compétente et une juge réfléchie. Marie a passé les vingt-cinq dernières années à juger des concours à travers l'Europe, gagnant une reconnaissance pour sa capacité à identifier des qualités subtiles—l'équilibre, le timing et la communication silencieuse entre le chien et le conducteur. Elle croit que les meilleurs chiens de berger ne sont pas simplement obéissants mais des partenaires intelligents capables de lire le bétail et le terrain. Marie a siégé dans des jurys internationaux et a contribué à façonner les normes modernes des concours qui honorent les méthodes de travail traditionnelles. Son propre programme d'élevage se concentre sur le tempérament et l'intelligence, produisant des chiens connus pour leur confiance calme et leur fiabilité. Marie aborde le jugement avec respect pour les compétiteurs et les moutons, comprenant que chaque passage représente des années de dévouement. Elle est honorée de contribuer son expertise à ce championnat.
\color{black}

\newpage
\section*{William O'Connor}
\nopagebreak

\begin{wrapfigure}{r}{0.35\textwidth}
  \centering
  \includegraphics[width=0.33\textwidth]{Placeholders/ab87043f15a845cd1608e3085d710ce4.jpg}
\end{wrapfigure}
\color{red}
William O'Connor est issu d'une longue lignée de bergers irlandais et a travaillé avec des border collies toute sa vie. Il a commencé à juger des concours il y a quinze ans et est connu pour son œil aiguisé pour les capacités naturelles et le style de travail solide. William apprécie les chiens qui font preuve d'initiative sans perdre le contact avec leur conducteur, et il apprécie les ajustements subtils qui séparent un travail compétent d'une véritable maîtrise. Son approche du jugement met l'accent sur l'équité et la clarté, garantissant que les conducteurs comprennent ce qui est récompensé et pourquoi. William a jugé des championnats dans toute l'Irlande, au Royaume-Uni et à travers l'Europe continentale. Il gère une ferme en activité où les chiens de berger sont des partenaires quotidiens essentiels, pas des animaux de démonstration. Cette perspective pratique informe sa philosophie de jugement : les concours devraient célébrer les compétences qui rendent possible le véritable travail de ferme. William recherche des chiens avec du courage, de l'intelligence et la rare capacité de rester calme sous pression. Il considère comme un privilège de juger ce championnat et d'assister au partenariat exceptionnel entre ces conducteurs et leurs chiens.
\color{black}
