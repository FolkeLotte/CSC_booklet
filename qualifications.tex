\chapter{Description des Compétitions}
\markboth{Compétitions}{Compétitions}
\section*{Qualifications}
19 pays du continent européen seront présents en août 2026, 
lors du CSC organisé par la fédération belge BSDS. 
Selon un ranking déterminé par la CSC (basé sur les résultats 
aux championnats continentaux antérieurs), chaque pays a organisé 
au préalable des concours sélectifs et a désigné son équipe nationale. 
A noter que seuls les border collie qui ont un pédigrée de travail 
peuvent participer aux concours spé-borders.

Du 27 au 30 août 2026, les compétiteur(trice)s s’affrontent 
pendant les 3 jours de qualifications. Chaque jour, les six meilleures 
combinaisons chien/maître obtiennent leur ticket pour la finale, 
qui a lieu le dimanche 30 août. 

Les runs de qualification sont des épreuves standardisées, 
cotées sur un total de 110 points.  Chaque run a une durée limite 
de 15’. Les runs (parcours) testent la complicité du chien et du 
maître en simulant le travail au quotidien du métier de berger : 
outrun (recherche), drive (conduite), shedding (séparation), 
pen (parc), single (isolement d’une brebis). 

C’est l’habileté commune à conduire, du binôme maitre/chien, qui 
est jugée (ici un lot de 5 brebis sur un parcours donné) 
Chaque écart de trajectoire, chaque changement de rythme, chaque 
relance de l’ordre donné au chien sont sanctionnés. La précision, 
la régularité, le respect des obstacles, l’obéissance déterminent 
la réussite d’un bon run

\section*{Finale}
\begin{wrapfigure}{r}{0.35\textwidth}
  \centering
  \includegraphics[width=0.4\textwidth]{OtherPictures/DoubleGatherPlaceholder.jpg}
\end{wrapfigure}
Les 18 finalistes se retrouvent en compétition pour la 
finale double recherche. 2 lots de 10 brebis sont placés 
successivement sur le terrain, idéalement hors vue du chien 
et à des endroits opposés. 
Le chien effectue 2 outruns (recherches) à la suite, 
abandonnant un lot pour aller chercher l’autre et un drive 
avec l’ensemble des brebis regroupées. Un tri de 5 brebis 
marquées, à isoler du lot de 20 brebis, est imposé aux 
concurrents, avant l’entrée au pen. La cotation des juges 
s’effectuant cette fois sur un total de 170 points et dans 
un délai de 30’. Les critères de réussite restent l’habilité 
de conduite sur le parcours, la précision des lignes, le 
contrôle du déplacement des brebis, l’obéissance aux ordres 
donnés. 

Le(la) champion(ne) d’Europe 2026 sera désigné(e) à l’issue 
de la finale.

