\chapter{Description des Compétitions - Competitions Description - Beschrijving van de Competities}
\markboth{Compétitions}{Compétitions}
\section*{Qualifications}

\textbf{Français:}

19 pays du continent européen seront présents en août 2026, 
lors du CSC organisé par la fédération belge BSDS. 
Selon un ranking déterminé par la CSC (basé sur les résultats 
aux championnats continentaux antérieurs), chaque pays a organisé 
au préalable des concours sélectifs et a désigné son équipe nationale. 
A noter que seuls les border collie qui ont un pédigrée de travail 
peuvent participer aux concours spé-borders.

Du 27 au 30 août 2026, les compétiteur(trice)s s'affrontent 
pendant les 3 jours de qualifications. Chaque jour, les six meilleures 
combinaisons chien/maître obtiennent leur ticket pour la finale, 
qui a lieu le dimanche 30 août. 

Les runs de qualification sont des épreuves standardisées, 
cotées sur un total de 110 points.  Chaque run a une durée limite 
de 15'. Les runs (parcours) testent la complicité du chien et du 
maître en simulant le travail au quotidien du métier de berger : 
outrun (recherche), drive (conduite), shedding (séparation), 
pen (parc), single (isolement d'une brebis). 

C'est l'habileté commune à conduire, du binôme maitre/chien, qui 
est jugée (ici un lot de 5 brebis sur un parcours donné) 
Chaque écart de trajectoire, chaque changement de rythme, chaque 
relance de l'ordre donné au chien sont sanctionnés. La précision, 
la régularité, le respect des obstacles, l'obéissance déterminent 
la réussite d'un bon run.

\vspace{1em}
\textbf{English:}

19 countries from the European continent will be present in August 2026, 
at the CSC organized by the Belgian federation BSDS. 
According to a ranking determined by the CSC (based on results 
from previous continental championships), each country has organized 
preliminary selective trials and designated its national team. 
Note that only border collies with a working pedigree 
can participate in specialist border collie trials.

From August 27 to 30, 2026, the competitors will compete 
during the 3 days of qualifications. Each day, the six best 
dog/handler combinations earn their ticket to the final, 
which takes place on Sunday, August 30.

The qualification runs are standardized tests, 
scored out of a total of 110 points. Each run has a time limit 
of 15 minutes. The runs (courses) test the rapport between dog and 
handler by simulating the daily work of a shepherd: 
outrun (gather), drive, shedding (separation), 
pen, single (singling out one sheep).

It is the combined handling ability of the handler/dog team that 
is judged (here with a group of 5 sheep on a given course). 
Every deviation from the trajectory, every change of pace, every 
repeated command given to the dog are penalized. Precision, 
consistency, respect for obstacles, and obedience determine 
the success of a good run.

\vspace{1em}
\textbf{Nederlands:}

19 landen van het Europese continent zullen aanwezig zijn in augustus 2026, 
tijdens het CSC georganiseerd door de Belgische federatie BSDS. 
Volgens een ranking vastgesteld door de CSC (gebaseerd op de resultaten 
van vorige continentale kampioenschappen), heeft elk land 
vooraf selectiewedstrijden georganiseerd en zijn nationale team aangewezen. 
Merk op dat alleen border collies met een werkstamboom 
kunnen deelnemen aan gespecialiseerde border collie-wedstrijden.

Van 27 tot 30 augustus 2026 strijden de deelnemers 
tijdens de 3 dagen van kwalificaties. Elke dag verdienen de zes beste 
hond/geleider-combinaties hun ticket voor de finale, 
die plaatsvindt op zondag 30 augustus.

De kwalificatieruns zijn gestandaardiseerde proeven, 
beoordeeld op een totaal van 110 punten. Elke run heeft een tijdslimiet 
van 15 minuten. De runs (parcours) testen de samenwerking tussen hond en 
geleider door het dagelijkse werk van een herder te simuleren: 
outrun (verzamelen), drive (drijven), shedding (scheiden), 
pen (hok), single (het afzonderen van één schaap).

Het is het gezamenlijke geleidingsvermogen van het geleider/hond-team dat 
beoordeeld wordt (hier met een groep van 5 schapen op een bepaald parcours). 
Elke afwijking van het traject, elke verandering van tempo, elk 
herhaald commando aan de hond worden bestraft. Precisie, 
regelmaat, respect voor obstakels en gehoorzaamheid bepalen 
het succes van een goede run.

\section*{Finale}
\begin{wrapfigure}{r}{0.35\textwidth}
  \centering
  \includegraphics[width=0.4\textwidth]{OtherPictures/DoubleGatherPlaceholder.jpg}
\end{wrapfigure}

\textbf{Français:}

Les 18 finalistes se retrouvent en compétition pour la 
finale double recherche. 2 lots de 10 brebis sont placés 
successivement sur le terrain, idéalement hors vue du chien 
et à des endroits opposés. 
Le chien effectue 2 outruns (recherches) à la suite, 
abandonnant un lot pour aller chercher l'autre et un drive 
avec l'ensemble des brebis regroupées. Un tri de 5 brebis 
marquées, à isoler du lot de 20 brebis, est imposé aux 
concurrents, avant l'entrée au pen. La cotation des juges 
s'effectuant cette fois sur un total de 170 points et dans 
un délai de 30'. Les critères de réussite restent l'habilité 
de conduite sur le parcours, la précision des lignes, le 
contrôle du déplacement des brebis, l'obéissance aux ordres 
donnés. 

Le(la) champion(ne) d'Europe 2026 sera désigné(e) à l'issue 
de la finale.

\vspace{1em}
\textbf{English:}

The 18 finalists compete in the double gather final. 2 groups of 10 sheep 
are placed successively on the field, ideally out of sight of the dog 
and at opposite locations. 
The dog performs 2 outruns (gathers) in succession, 
leaving one group to go fetch the other and a drive 
with all the sheep gathered together. A shed of 5 marked sheep, 
to be separated from the group of 20 sheep, is required of 
the competitors, before entering the pen. The judges' scoring 
this time is out of a total of 170 points and within 
a time limit of 30 minutes. The criteria for success remain 
the handling skill on the course, the precision of the lines, 
the control of the sheep's movement, and obedience to the commands 
given.

The European Champion 2026 will be designated at the end of the final.

\vspace{1em}
\textbf{Nederlands:}

De 18 finalisten strijden in de dubbele verzamel finale. 2 groepen van 10 schapen 
worden achtereenvolgens op het veld geplaatst, idealiter buiten het zicht van de hond 
en op tegenovergestelde locaties. 
De hond voert 2 outruns (verzamelingen) achter elkaar uit, 
waarbij hij de ene groep verlaat om de andere te halen en een drive 
met alle verzamelde schapen. Een shed van 5 gemarkeerde schapen, 
die moeten worden gescheiden van de groep van 20 schapen, wordt van 
de deelnemers geëist, vóór het binnengaan in de pen. De beoordeling door de 
juryleden gebeurt dit keer op een totaal van 170 punten en binnen 
een tijdslimiet van 30 minuten. De criteria voor succes blijven 
het geleidingsvermogen op het parcours, de precisie van de lijnen, 
de controle over de beweging van de schapen en de gehoorzaamheid aan de 
gegeven commando's.

De Europese Kampioen 2026 zal worden aangewezen aan het einde van de finale.
